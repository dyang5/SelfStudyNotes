\documentclass[11pt]{article}
\usepackage{styletemplate}

\begin{document}

\title{\LARGE \textbf{Math 356: Number Theory (Rutgers)}}
\date{Winter Break, 2023}
\author{\textsc{David Yang}}

\maketitle

\begin{abstract}
These notes arise from lecture videos of Math 356: Number Theory (Quadratic Forms), originally taught by 
Professor \href{https://sites.math.rutgers.edu/~alexk/}{Alex Kontorovich}, at Rutgers University. I am grateful
to Professor Kontorovich for releasing the \href{https://www.youtube.com/playlist?list=PLs6rMe3K87LEwc77iVba5AsAUPC1qVowy}{lecture videos} online. 
I am responsible for all faults in this document, mathematical or otherwise; any merits of the
material here should be credited to Professor Kontorovich and not to me.
Feel free to message me with any suggestions or corrections at \href{mailto:dyang5@swarthmore.edu}{dyang5@swarthmore.edu}.
\end{abstract}

\tableofcontents

\newpage

\section{Introduction to Fermat's Last Theorem}

\begin{theorem}[Fermat's Last Theorem, 1637]
There are no positive integers $x$, $y$, $z$ that satisfy \[x^n + y^n = z^n\]
for integer $n > 2$. \\

An equivalent formulation is the one discussed in lecture:
\[
    x^n + y^n = z^n
\] where $n > 2$ is an integer, has no non-trivial solutions $x$, $y$, and $z$ in $\Q, \Z$.
\end{theorem}

\begin{definition}[Diophantine Problem]
A \textbf{Diophantine problem} is a polynomial equation solved in $\Z$ and $\Q$. \\

\textit{Fermat's Last Theorem is an example of a Diophantine problem.}
\end{definition}

\textit{Historically, in 1993, Andrew Wiles published a proof of Fermat's Last Theorem. Later, Wiles and his student Richard Taylor rectified the 1993 proof and published
the first successful proof. For these efforts, Wiles won the 2016 Abel Prize and
the 2017 Copley Medal.} \\

With some extra background, we can prove Fermat's Last Theorem for $n = 4$.

\subsection{Parameterization of Pythagorean Triples}

\begin{theorem}[Primitive Pythagorean Triple Generation]
If $(x, y, z)$ is a \textbf{primitive} pythagorean triple\footnote{if $(x, y, z)$ is a primitive Pythagorean triple, then it follows that $x$, $y$, and $z$ are pairwise primtiive.    
} ($\nexists \, d$ such that $d \mid x$, $d\mid y$ and $d \mid z$), then $z$ is odd. Furthermore, assuming that
$x$ is odd and $y$ is even, then there exists coprime $r, s \in \Z$ with such that
\[ x = r^2 - s^2, \, y = 2rs, \, z = r^2 + s^2. \]
\end{theorem}

\begin{proof}[Proof Sketch (Theorem 3)]
Since $y$ is even, we can rewrite $y = 2y_1$ for some integer $y_1$. Since $(x, y, z)$ are a Pythagorean triple, it follows that
\( y^2 = z^2 - x^2 = (z-x)(z+x). \) Substituting and expanding, we get that 
\[
    4y_1^2 = (z-x)(z+x).
\] 
Since $x$, $z$ are both odd, we can rewrite $z-x = 2a$ and $z+x =2b$, so 
\[ y_1^2 = ab\] where $(a, b) = 1$. Consequently, $a$, $b$ must themselves be perfect squares. We conclude that $y_1=rs$, where $r^2 = a$, $s^2=b$, and so it follows that
$y = 2rs$ for coprime $r$, $s$. 
\end{proof}

\subsection{Pythagorean Varieties}

\begin{definition}[Variety]
A \textbf{variety} is a set of solutions to a system of polynomial equations.
\end{definition}

Consider the variety $\mathcal{V}: x^2 + y^2 - z^2$, which is a Pythagorean variety. Note that
$\mathcal{V}(\R)$ is a structure consisting of two cones which are reflections about the $z$-axis,
and $ \mathcal{V}(\Z)$, $\mathcal{V}(\Q)$ are subsets of $\mathcal{V} (\R)$.

\begin{lemma*}
For every $\vec{v} \in \mathcal{V}(\Q)$, there exists a unique primitive $\vec{w} \in \mathcal{V}(\Z)_+^{0}$ with $\vec{w} \sim \vec{v}$, where
$\vec{u} \sim \vec{v} \iff \vec{u} = \lambda\vec{v}, \, \lambda \in \R \setminus \{ 0\}$.
\end{lemma*}

Note that $\mathcal{V}(\R)_+ / \sim$ can be thought of as $S^1$; for a point
$(x, y, z) \in \mathcal{V} (\R)_+ \mapsto \left( \frac{x}{z}, \frac{y}{z}, 1 \right)$. Furthermore, $\mathcal{V}(\R) / \sim = S^1 \cup \{0\}.$ \\

Furthermore, note that another set of representatives of $\mathcal{V} (\Q)_+ / \sim$ is $S^1(\Q)$. This gives a notion for why Pythagorean triples are fundamental:
they parameterize rational points on the unit circle. Formally, 
\[
    \left( \frac{r}{u}, \frac{t}{u} \right) \in S^1(\Q) \iff (r, t, u) \in \mathcal{V} (\Z)_+^{0} 
\]

\subsection{Proof of Fermat's Last Theorem for $n=4$}

\textbf{Statement.} $x^4 + y^4 = z^4$ in $\Z$ implies $xyz = 0$. \\
\textbf{Stronger Statement.} $x^4 + y^4 = z^2$ in $\Z$ implies $xyz = 0$.
\begin{proof}
Suppose there is a solution to $x^4 + y^4 = z^2$ in $\Z$ with $xyz \neq 0$. Then $(x^2, y^2, z)$ is a Pythagorean triple. Furthermore, since we can scale Pythagorean triples by their GCD to get a primitive triple, we can assume that 
$(x^2, y^2, z) \in \mathcal{V}(\Z)_+^0$ is rather a primitive Pythagorean triple. By the Parameterization
of primitive Pythagorean triples, there exists coprime $r, s \in \Z$ such that 
\[
  x^2 = r^2 - s^2, \, y^2 = 2rs, \, z = r^2 + s^2.  
\]

Furthermore, since $x^2$ is odd and $y^2$ is even by assumption, $r$ and $s$ also have opposite parity. From the first equation, we know that 
\[
    x^2 + s^2 = r^2,
\] where $x$ is odd, forcing $s$ to be even, and $r$ to be odd. Once again, $(x, s, r)$ is a primitive Pythagorean triple, so there exists coprime $u$, $v \in \Z$ such that
\[ 
    x = u^2 - v^2, \, s = 2uv, \, r = u^2 + v^2 
\]
where $(r, s) = 1$. We also know that $y^2 = 2rs$ from our first primitive Pythagorean triple $(x^2, y^2, z)$. Since $y$ is even, we can express it as $y=2y_1$ for some $y_1 \in \Z$. 
Since $r = u^2 + v^2$, $s = 2uv$, and $y=2y_1$, we have that
\[
    4y_1^2 = 2(u^2 + v^2)(2uv)
\]
    
and so 
\[
    y_1^2 = uv(u^2 + v^2).    
\]

Note however that $(u, v) = 1$. Consequently, $u$, $v$, and $u^2 + v^2$ are three pairwise coprime numbers whose product is a perfect square. Thus, $u$, $v$, and $u^2 + v^2$ must themselves be a perfect square.
We now write 
\[
    u = a^2, \, v = b^2, \, u^2 + v^2 = c^2
\]

From the third equation, we have that $a^4 + b^4 = c^2$. \\

Note that if $xyz \neq 0$, then
\[ c \leq c^2 = u^2 + v^2 \leq (u^2 + v^2)^2 = r^2 < r^2 + s^2 = z. \]

Thus, we have found a smaller, in the sense that $c < z$, primitive solution to $x^4 + y^4 = z^2$ with $abc \neq 0$. \\

We can continue this process of finding primitive solutions until we arive at a trivial solution. Thus, by the principle of infinite descent\footnote{or Fermat's descent}, we conclude
that $x^4 + y^4 = z^2$ in $\Z$ implies $xyz = 0$. \end{proof}

\newpage

\section{Modular Arithmetic}

\textit{Note: Basic modular arithmetic notes skipped due to prior knowledge. Important theorems/results added for completeness.}

\begin{theorem}[Fermat's Little Theorem]
If $p$ is a prime with $p \nmid a$, 
\[
    a^{p-1} \equiv 1 \pmod p.
\]
\end{theorem}

\begin{proof}
Note that $\{ 1\cdot a, 2\cdot a, \dots, (p-1)\cdot a, p\cdot a\}$
is a complete residue system modulo $p$. It follows that
\[
    (1\cdot a)(2\cdot a) \dots ((p-1) \cdot a) \equiv 1\cdot 2\cdot 3 \dots (p-1) \pmod p.
\]

Simplifying, we have that
\[
    (p-1)! a^{p-1} \equiv (p-1) \pmod p.
\]

Equivalently, we have that $p \mid (p-1)! \left( a^{p-1} - 1 \right).$ Since $p \nmid (p-1)!$, it follows that $p \mid a^{p-1} - 1$, so 
\[
    a^{p-1} \equiv 1 \pmod p. \qedhere
\]
\end{proof}

\subsection{Linear Diophantine Equations (and Euclidean Algorithm)}
Let $S = \{  an + bm \mid n, m \in \Z \}$\dots
\begin{theorem}
$S = d\Z$, where $d = \mathrm{gcd}(n, m).$  
\end{theorem}

\begin{proof}
First, note that $S \subseteq d\Z$. If $d \mid n$ and $d \mid m$, then $d \mid an$ and $d \mid bm$, so $d \mid an + bm$. 
It remains to show that $an + bm = d$ has a solution. It is possible to find such a solution by reversing the steps of the Euclidean Algorithm using $n$ and $m$.
\end{proof}

\begin{theorem}[Euclidean Algorithm]
The \textbf{Euclidean Algorithm} can be used to find the GCD of two integers $n > m > 0$ as follows:
\begin{itemize}
    \item Apply the `division' algorithm to rewrite $n = m \cdot q_1 + r_1$, where $0 \leq r_1 < m$.\footnote{for reference, we can express $n$ as $r_{-1}$ and $m$ as $r_0$.}
    \item Continue this process, finding $q_j, r_j$ such that $r_{j-2} = r_{j-1} q_j + r_j$, with $0 \leq r_j < r_{j-1}$.
    \item Stop at some finite $J \leq m$ with $r_J = 0$.
\end{itemize}

It follows that $r_{J-1} = \mathrm{gcd}(n, m).$ 
\end{theorem}

\begin{proof}
First, let's show that $\mathrm{gcd}(n, m) \mid r_{J-1}.$ Note that if $l \mid n$ and $l \mid n$, $l \mid n - mq_1$, so $l \mid r_1$. Similarly, if 
$l \mid r_{j-2}$ and $l \mid r_{j-1}$, then $l \mid r_{j-2} - r_{j-1}q_j$, so $l \mid r_j$. Thus, it follows, that $l \mid r_{J-1},$ and so $\mathrm{gcd}(n, m) \mid r_{J-1}.$ \\

It remains to show that $r_{J-1} \mid \mathrm{gcd}(n, m)$. Note that
$r_{J-2} = r_{J-1}q_{J} + r_J$, and $r_J = 0$. It follows that $r_{J-1} \mid r_{J-2}$. Following our steps backwards, 
we will find that $r_{J-1} \mid r_j$ for all $j \in [-1, J-2].$ 
Recall that $r_0 = m$ and $r_{-1} = n$. Consequently, since $r_{J-1} \mid r_j$ for all $j$, we have that $r_{J-1} \mid n$ and $r_{J-1} \mid m$, so $r_{J-1} \mid \mathrm{gcd}(n, m).$ \\

Since $\mathrm{gcd}(n, m) \mid r_{J-1}$ and $r_{J-1} \mid \mathrm{gcd}(n, m)$, it follows that $r_{J-1} = \mathrm{gcd}(n, m)$, as desired. 
\end{proof}

\subsection{Connections to Algebra}
When we studied Linear Diophantine Equations, we looked at sets $S = \{ xn + ym \mid x, y \in \Z \}.$ Such a set $S$ is an example of an ideal.
\begin{definition}[Ideals]
An \textbf{ideal} of the ring $\Z$ satisfies
\begin{itemize}
    \item For $z \in S$, $rz \in S$ for all $r \in \Z$.
    \item For $z_1, z_2 \in S$, $z_1 \pm z_2 \in S$.
\end{itemize}
\end{definition} 

\begin{definition}
An ideal $S$ of the ring $\Z$ is \textbf{principal} if $\exists \, d \in \Z$ such that
\[ S = (d) = \{ d \cdot r \mid r \in \Z \} = d\Z. \]     
\end{definition}

\begin{theorem}
$\Z$ is a \textbf{Principal Ideal Domain} (every ideal is principal). \\

Equivalently, suppose that $S = (n_1, \dots, n_k)$ is an ideal. Then $\exists \, d \in \Z$ such that $S = (d)$.\footnote{in fact, $d = \mathrm{gcd}(n_1, \dots, n_k)$. }
\end{theorem}

\begin{proof}
Suppose that $S = (n_1, \dots, n_k).$ If $l \mid n_1, \dots, l \mid n_k$, then for all $z \in S, l \mid z$. 
Thus, $d = \mathrm{gcd}(n_1, \dots, n_k) \mid z$ for all $z \in S$, so $S \subseteq d\Z$. \\

It remains to show that $d = \mathrm{gcd}(n_1, \dots, n_k) \in S.$ Let $r$ be the smallest positive element in $S$. Clearly, $r\Z \subseteq S$. 
It follows, by the Euclidean Algorithm that $d \mid r$, so $d = r$. Thus, $d\Z \subseteq S$.
\end{proof}

\begin{definition}[(Loose Definitions of) Groups, Rings, and Fields]
A \textbf{group} is a set $S$ with an operation $+$, $0$, and inverses. \\

A \textbf{ring} is a set $S$ with operations $+, \times$ that supports addition, subtraction, and multiplication. \\

A \textbf{field} is a set $S$ with operations $+, \times$ that supports addition, subtraction, multiplication, and divison\footnote{for all $x \in S \setminus \{0\}$, $\exists \, y \in S$ with $xy = 1$.}
\end{definition}

\begin{theorem}
($\Z / n\Z, +, \times$) is a field if and only if $n$ is prime.
\end{theorem}

\begin{proof}
For prime $p$ and $a \not\equiv 0 \pmod p$, $a^{p-1} \equiv 1 \pmod p$ by Fermat's Little Theorem, so $a^{p-2} = \overline{a}$, the modular inverse of $a \pmod p.$ Consequently, we have
found an explicit inverse for each $a \not\equiv 0 \pmod p$. \\

On the other hand, suppose that $n$ is not prime and let $a \in \Z / n\Z$ with $a \not\equiv 0 \pmod n$ and $\mathrm{gcd}(a, n) > 1$. Then $ab + nm = 1$ has no solutions, so $ab \equiv 1 \pmod n$ has no solutions $b$.  
\end{proof}

\begin{definition}[Units]
The units of $\Z / n\Z$, expressed as $(\Z / n\Z)^{\times}$, is
\[
    (\Z / n\Z)^{\times} = \{ a \pmod{n} \mid \exists \,b : ab = 1 \}.
\]
\end{definition}
\end{document}