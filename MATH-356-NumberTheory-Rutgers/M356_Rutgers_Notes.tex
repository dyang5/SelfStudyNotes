\documentclass[11pt]{article}
\usepackage{styletemplate}

\begin{document}

\title{\LARGE \textbf{Math 356: Number Theory (Rutgers)}}
\date{Winter Break, 2023}
\author{\textsc{David Yang}}

\maketitle

\begin{abstract}
These notes arise from lecture videos of Math 356: Number Theory (Quadratic Forms), originally taught by 
Professor \href{https://sites.math.rutgers.edu/~alexk/}{Alex Kontorovich}, at Rutgers University. I am grateful
to Professor Kontorovich for releasing the \href{https://www.youtube.com/playlist?list=PLs6rMe3K87LEwc77iVba5AsAUPC1qVowy}{lecture videos} online. 
I am responsible for all faults in this document, mathematical or otherwise; any merits of the
material here should be credited to the lecturer, not to me.
Feel free to message me with any suggestions or corrections at \href{mailto:dyang5@swarthmore.edu}{dyang5@swarthmore.edu}.
\end{abstract}

\tableofcontents

\newpage

\section{Introduction to Fermat's Last Theorem}

\begin{theorem}[Fermat's Last Theorem, 1637]
There are no positive integers $x$, $y$, $z$ that satisfy \[x^n + y^n = z^n\]
for integer $n > 2$. \\

An equivalent formulation is the one discussed in lecture:
\[
    x^n + y^n = z^n
\] where $n > 2$ is an integer, has no non-trivial solutions $x$, $y$, and $z$ in $\Q, \Z$.
\end{theorem}

\begin{definition}[Diophantine Problem]
A \textbf{Diophantine problem} is a polynomial equation solved in $\Z$ and $\Q$. \\

\textit{Fermat's Last Theorem is an example of a Diophantine problem.}
\end{definition}

\textit{Historically, in 1993, Andrew Wiles published a proof of Fermat's Last Theorem. Later, Wiles and his student Richard Taylor rectified the 1993 proof and published
the first successful proof. For these efforts, Wiles won the 2016 Abel Prize and
the 2017 Copley Medal.} \\

With some extra background, we can prove Fermat's Last Theorem for $n = 4$.

\subsection{Parameterization of Pythagorean Triples}

\begin{theorem}[Primitive Pythagorean Triple Generation]
If $(x, y, z)$ is a \textbf{primitive} pythagorean triple\footnote{if $(x, y, z)$ is a primitive Pythagorean triple, then it follows that $x$, $y$, and $z$ are pairwise primtiive.    
} ($\nexists \, d$ such that $d \mid x$, $d\mid y$ and $d \mid z$), then $z$ is odd. Furthermore, assuming that
$x$ is odd and $y$ is even, then there exists coprime $r, s \in \Z$ with such that
\[ x = r^2 - s^2, \, y = 2rs, \, z = r^2 + s^2. \]
\end{theorem}

\begin{proof}[Proof Sketch (Theorem 3)]
Since $y$ is even, we can rewrite $y = 2y_1$ for some integer $y_1$. Since $(x, y, z)$ are a Pythagorean triple, it follows that
\( y^2 = z^2 - x^2 = (z-x)(z+x). \) Substituting and expanding, we get that 
\[
    4y_1^2 = (z-x)(z+x).
\] 
Since $x$, $z$ are both odd, we can rewrite $z-x = 2a$ and $z+x =2b$, so 
\[ y_1^2 = ab\] where $(a, b) = 1$. Consequently, $a$, $b$ must themselves be perfect squares. We conclude that $y_1=rs$, where $r^2 = a$, $s^2=b$, and so it follows that
$y = 2rs$ for coprime $r$, $s$. 
\end{proof}

\end{document}